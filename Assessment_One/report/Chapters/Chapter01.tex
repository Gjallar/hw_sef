%************************************************
\chapter{Introduction}\label{ch:introduction}
%************************************************

This report shall provide detailed information about the implementation of the cabin manager application.

\section{Attributes}
\label{sec:attributes}

Apart from the mandatory attributes, the following attributes were chosen to be implemented as well:
\begin{description}
\item[Size:] The area of available space in the cabin.
\item[Condition:] The condition the cabin is in. Only the following values are allowed: \texttt{PERFECT}, \texttt{GOOD}, \texttt{FAIR}, \texttt{BAD}, \texttt{IN\_SHAMBLES}, \texttt{UNKNOWN}. The limitation to these attributes is achieved by utilizing an \texttt{enumeration}.
\end{description}

\section{Cost calculation}
\label{sec:cost_calculation}

The calculation of a cabin's cost for one night, are calculated according to the following formula:

\texttt{BASIC\_COST + CONDITION\_COST + FACILITIES\_COST + SIZE\_COST + (BED\_TO\_ROOM\_RATIO * BED\_TO\_ROOM\_MULTIPLIER)}

\texttt{BASIC\_COST} and \texttt{BED\_TO\_ROOM\_MULTIPLIER} are constants that can be set in the \texttt{cabin} class.

\section{Frequency reports}
\label{sec:frequency_report}

The frequency-report provided outputs the number of cabins of a certain condition.

E.g. if two cabins are of the condition ``\texttt{IN\_SHAMBLES}'' and one is of the condition ``\texttt{GOOD}'' the output would be as follows:

\begin{tabular}{c|c|c|c|c}
BAD & FAIR & GOOD & IN\_SHAMBLES & PERFECT \\ 
0 & 0 & 1 & 2 & 0 \\ 
\end{tabular}

\section{Status report}
\label{sec:status_report}

Even though not well-designed the application should meet the specification fully as all requirements were implemented and tested to function even in the case of incorrect input.
